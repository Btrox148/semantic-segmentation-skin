\documentclass{article}
\begin{document}
\begin{center} {\scshape\LARGE Proyecto de tesis de licenciatura \par} \end{center}
Titulo tentativo   ``Detección de melanoma mediante segmentación semántica"

\section*{Justificación}
Para determinar que especies existen en un sector forestal no basta con conocer las ya existentes sino que se debe realizar una inventario forestal para poder determinar las especies predominantes en la zona. Sin embargo, las técnicas tradicionales para generar un inventario forestal suelen ser lentas y a veces, anticuadas.\newline\\
Esta tesis muestra como haciendo uso del procesamiento de imágenes se puede desarrollar una solución eficiente que permita reducir tiempos y costos al generar un inventario forestal en comparación a las técnicas tradicionales para generar inventarios forestales.

\section*{Hipótesis}
Se puede lograr una solución eficiente e innovadora enfocando el procesamiento de imágenes a la generación de inventarios forestales haciendo uso del aprendizaje máquina y la visión computacional.

\section*{Objetivo}
Realizar un software que permita a las personas encargadas de analizar zonas forestales, generar inventarios forestales de forma eficiente, reduciendo el tiempo invertido en las técnicas tradicionales y cambiándolas a técnicas de procesamiento de imágenes basadas en aprendizaje máquina.

\section*{Metodología}
El desarrollo de la tesis está seccionada en:

\begin{description}
\item[Investigación]{Hace un análisis del problema que se pretende resolver, define el objetivo general y específicos que deben cumplir la tesis.}

\item[Implementación]{Desarrolla e implementa la solución, utilizándola sobre contextos en los que funcionaría la solución desarrollada. }

\item[Evaluación]{Diseña y prueba experimentos para analizar el comportamiento de la solución propuesta en distintos contextos.}

\end{description}
\pagenumbering{gobble}
\end{document}
