%Resumen

\chapter{Resumen}
\markboth{Resumen}{}

{\renewcommand{\baselinestretch}{1.1}\selectfont
{\setlength{\leftskip}{10mm}
\setlength{\parindent}{-10mm}

\autor.

Candidato para obtener el grado de \grado\orientacion.

\uanl.

\fime.

Título del estudio: \textsc{\titulo}.

\noindent Número de páginas: \pageref*{lastpage}.}

%%% Comienza a llenar aquí
\paragraph{Objetivos y método de estudio:}
El objetivo de el presente trabajo de tesis el uso de la tecnología de redes neuronales profundas (\emph{deep learning}) para la detección de melanoma de piel, mediante el uso de DL es posible desarrollar modelos cuyo entrenamiento permite obtener la secuencia específica para transformar el dato de entrada en la dimensión y forma deseada. El dato de entrada en este caso consta de tres matríces de intensidad de pixeles que corresponden a los tres canales de la imágen a color, y el dato de salida es una matríz de un solo canal correspondiente al mapa de segmentación en el cuál se remarcan las regiones de la imagen que corresponden a la categoría clasificada (tejido sano, melanoma). Para obtener el modelo que realiza la secuencia de transformaciones, se realiza un \emph{entrenamiento}, esto se refiere a un proceso de evaluación y optimización en el cuál se introduce una imagen de entrada por el modelo inicializado con valores aleatorios o pre-entrenados. Al obtener un dato de salida, ésta se evalúa con la salida conocida y se determina la precisión del modelo, posteriormente se ajustan todos los parámetros del modelo para acercarse a la salida conocida.  

\paragraph{Contribuciones y conclusiones:}
La principal contribución de este trabajo de tesis está dirigido al área de \emph{hospital 4.0}, ya que el método planteado para la detección del cáncer en la piel involucra el úso de las tecnologías desarroladas mediante inteligencia artificial. Esto beneficia especialmente al proceso el paciente puede obtener su diagnóstico.
\coltext{Agregar mas información.}


\bigskip\noindent\begin{tabular}{lc}
\vspace*{-2mm}\hspace*{-2mm}Firma de la asesora: & \\
\cline{2-2} & \hspace*{1em}\asesor\hspace*{1em}
\end{tabular}}

