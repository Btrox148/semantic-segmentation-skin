%Resumen

\chapter{Resumen}
\markboth{Resumen}{}

{\renewcommand{\baselinestretch}{1.1}\selectfont
{\setlength{\leftskip}{10mm}
\setlength{\parindent}{-10mm}

\autor.

Candidato para obtener el grado de \grado\orientacion.

\uanl.

\fime.

Título del estudio: \textsc{\titulo}.

\noindent Número de páginas: \pageref*{lastpage}.}

%%% Comienza a llenar aquí
\paragraph{Objetivos y método de estudio:}
Desarrollar una herramienta de asistencia para la detección de cáncer de piel utilizando las técnicas mas actuales de visión computacional e inteligencia artificial, se pretende desarrollar mediante \emph{software} y tecnicas de \emph{\gls{seg}} una aplicación que permita introducir una imagen y como resultado obtengamos un mapa de características segmentado en una o más categorías. 

\paragraph{Contribuciones y conclusiones:}
La principal contribución de este trabajo de tesis está dirigido al área de \emph{hospital 4.0}, ya que el método planteado para la detección del cáncer en la piel involucra el úso de las tecnologías desarroladas mediante inteligencia artifica. Esto beneficia especialmente al proceso el paciente puede obtener su diagnostico.
\coltext{Agregar mas información.}


\bigskip\noindent\begin{tabular}{lc}
\vspace*{-2mm}\hspace*{-2mm}Firma de la asesora: & \\
\cline{2-2} & \hspace*{1em}\asesor\hspace*{1em}
\end{tabular}}

