%Resumen

\chapter{Resumen}
\markboth{Resumen}{}

{\renewcommand{\baselinestretch}{1.1}\selectfont
{\setlength{\leftskip}{10mm}
\setlength{\parindent}{-10mm}

\autor.

Candidato para obtener el grado de \grado\orientacion.

\uanl.

\fime.

Título del estudio: \textsc{\titulo}.

\noindent Número de páginas: \pageref*{lastpage}.}

%%% Comienza a llenar aquí
\paragraph{Objetivos y método de estudio:}
El presente trabajo de tesis habla sobre la aplicación del aprendizaje profundo o también denominado \emph{deep learning}, para la detección de melanoma de piel mediante la aplicación de las redes neuronales profundas.
El melanoma es un padecimiento que se origina en las células de la piel cuando los melanocitos (las células que dan el color marrón a la piel), comienzan a crecer sin control causando estragos a quien la padece y que sin una detección y tratamiento temprano, puede aumentar exponencialmente el riesgo de fallecimiento del afectado.
Una red neuronal es un modelo matemático que pretende simular el proceso de aprendizaje de las neuronas biológicas mediante el recibimiento de señales binarias o normalizadas, la asignación de un peso sináptico o importancia para cada señal y una función de activación que discrimine o realce dichas señales. Es posible desarrollar un modelo cuyo entrenamiento dé como resultado un sistema que ejecute una secuencia específica de transformaciones con el dato entrante, mediante la optimización automática de pesos sinápticos en las distintas capas del filtrado, para transformar la señal o dato entrante al valor deseado a la salida. El dato entrante, en el caso de imágenes a color, trata de tres matrices correspondientes a los tres canales de colores primarios (azul, verde, rojo), y la salida deseada trata de un mapa probabilístico correspondiente al mapa de segmentación en el cual se ilustran las regiones de la imagen mediante colores distintos, refiriéndose mediante su color a las categoría correspondiente de dicha región (tejido sano, tejido melanoma). Para obtener un modelo que realice dicha serie de transformaciones se requiere de dos procesos: \emph{reducción} y \emph{reconstrucción}. La fase de \emph{reducción} trata de reducir la dimensión de la imagen de entrada conservando su información característica, haciendo más fácil su computación en los procesos siguientes; y la fase de \emph{reconstrucción}, como el nombre indica, corresponde a la reconstrucción de la imagen basándose en una predicción aplicada al dato previamente codificado en la fase de \emph{reducción}. Para finalizar se realiza un escalado del mapa probabilístico obtenido de la predicción para crear la máscara de segmentación con sus correspondientes regiones categorizadas.

\paragraph{Contribuciones y conclusiones:}
La principal contribución del presente trabajo de tesis es la de establecer los pasos requeridos para llevar a cabo el proceso de implementación de redes neuronales profundas para realizar tareas de segmentación semántica. El lenguaje utilizado en esta implementación fue el de \emph{Python}, debido a que sus librerías permiten desplegar ágilmente modelos neuronales y utilizar métricas que permitan evaluar y experimentar su funcionamiento. Una de las métricas utilizadas para evaluar dicha hipótesis es la del índice de Jaccard, su función es comparar un mapa binario con un mapa probabilístico obtenido por el modelo generado y determinar la similitud entre ambos. 

Para la implementación se propuso el uso de la arquitectura \texttt{FPN}\footnote{Feature Pyramid Network}, la cuál es compatible con el codificador \texttt{ResNet}, ambos consisten en secuencias de convolución y deconvolución que realizarán las tareas de reducir las dimensiones de los datos en un solo dato multi-dimensional y de recrear el mapa probabilístico correspondiente a las computaciones del modelo. Para validar la selección de la arquitectura y codificador se estableció una serie de experimentos dependientes unos de otros en donde se determina la mejor configuración de parámetros para la creación y el entrenamiento del modelo así como la validación de las máscaras obtenidas mediante sobreposición de pixeles de ambas máscaras, mediante el coeficiente de dados y el criterio de Jaccard.

Mediante el experimento del filtro de umbral se obtuvieron datos sobre las características de las imágenes en cuanto la intensidad de los pixeles antes y después de aplicarse, y con los diferentes criterios de evaluación mencionados, se obtuvo un perfil de aprendizaje del modelo en donde se puede visualizar el principal comportamiento durante las épocas iniciales y las finales.

\bigskip\noindent\begin{tabular}{lc}
\vspace*{-2mm}\hspace*{-2mm}Firma de la asesora: & \\
\cline{2-2} & \hspace*{1em}\asesor\hspace*{1em}
\end{tabular}

