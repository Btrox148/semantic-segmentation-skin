
\chapter{Introducción}

La piel es considerada el órgano mas grande del cuerpo humano y está compuesta por tres capas: \emph{\gls{epidermis}}, \emph{\gls{dermis}} e \emph{\gls{hipodermis}}. Las principales funciones de la piel son protegernos de las hostilidades del medio ambiente tales como la radiación solar,  regular la temperatura, aislarnos de las bacterias en el medio ambiente y también cumple otras funciones tales como la absorción de vitamina D.

Sin embargo debido a las radiaciones de la luz solar, es muy común desarrollar anomalías que afectan la forma en que las células de nuestra piel se reproducen, causando graves daños a nuestra salud que en muchas ocasiones puede llegar a ser letal. Estas anomalías en la piel se denominan como \emph{cáncer}, y existen tres diferentes categorías de cáncer en la piel: Basalioma, Carcinoma y Melanoma.

En los últimos años se han logrado muchos avances en cuanto al desarrollo de software inteligentes, una de las tecnologías emergentes y que está tomando gran importancia es la <<\emph{\gls{rn}}>>, se trata de una tecnología que tiene la capacidad de aprender de datos históricos y crear un modelo matemático capaz de predecir, clasificar o recrear valores. Algunos de los sectores que han mostrado un incremento en el uso de ésta tecnología son: el sector automotriz (piloto automático), el sector de manufactura (optimización de procesos), el sector de entretenimiento (recomendaciones personalizadas), el sector médico (diagnóstico de imágenes). 

Este experimento tiene como objetivo la clasificación de tejidos sanos y tejidos con posible cáncer de piel (basalioma, carcinoma, melanoma) en imágenes, mediante el uso de la red neuronal de \emph{\gls{seg}} basada en el modelo propuesto en \citealt{wu2019fastfcn}, con la finalidad de asistir al médico especializado en el diagnóstico de cáncer de piel a brindar atención a los pacientes con mayor probabilidad de padecer la enfermedad.

\section{Hipótesis}
Es posible clasificar los píxeles en distintas categorías dentro de una imagen gracias a las tecnologías actuales de inteligencia artificial y las técnicas de segmentación. Mediante la técnica de \emph{\gls{seg}} es posible crear un reconocedor visual que no solo detecte la presencia y ubicación del elemento a reconocer, sino que, también obtenga otros datos descriptivos del elemento como el tamaño, forma y región que abarca dentro de la imagen. 

\section{Objetivos}
Primero en \emph{Objetivo General} se hablará de manera conceptual la problemática a resolver tales como cuales son las situaciones en las que podemos optimizar la resolución de un problema mediante el uso de la tecnología <<deep learning>>, posteriormente en los \emph{Objetivos Específicos} se hablará sobre el objetivo al que queremos llegar utilizando antecedentes de herramientas y teorías de otros experimentos similares para adaptarlo a nuestro problema.

\subsection{Objetivo General}
El \emph{objetivo general} es crear una herramienta capáz de asistir al personal medico a detectar posibles padecimientos de cáncer de piel utilizando las tecnologías mas actuales de inteligencia artificial denominada como <<deep learning>>.


\subsection{Objetivo Específico}

\section{Estructura de la tesis}

\chapter{Antecedentes}

\chapter{Estado del Arte}

\chapter{Descripción de los datos}

\chapter{Implementación de la Solución}

\chapter{Resultados}

