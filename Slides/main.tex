\documentclass{beamer}
\usepackage[utf8]{inputenc}
\usepackage[spanish]{babel}
\usepackage{blindtext}
\usepackage{ragged2e}

\setbeamertemplate{caption}[numbered]

\usetheme{Madrid}

%Information to be included in the title page:
\title{Detección de melanoma mediante segmentación semántica}
\author{Mario Alberto Flores Hernández}
\institute{ \\Universidad Autónoma de Nuevo León \\Facultad de Ingeniería Mecánica y Eléctrica}
\date{2020}


\begin{document}

\frame{\titlepage}

\begin{frame}
    \frametitle{Índice}
        \tableofcontents
\end{frame}
\section{Introducción}
\begin{frame}
    \frametitle{Introducción}
    La inteligencia artificial nos permite crear modelos que replican secuencias de transformaciones definidas por los datos entrantes.
\end{frame}

\subsection{Motivación}
\begin{frame}
    \frametitle{Motivacion}
    El melanoma de piel es un padecimiento que puede ser tratado cuando es detectado a tiempo, mediante la implementación de tecnologías de reconocimiento automático podrían aumentar las posibilidades de tratarse a tiempo.
\end{frame}

\subsection{Hipótesis}
\begin{frame}
    \frametitle{Hipótesis}
    Mediante la implementación de redes neuronales convolucionales es posible entrenar modelos que reconozcan mediante la técnica de segmentación semántica la presencia y región del melanoma dentro de imágenes dermatológicas.
\end{frame}

\subsection{Objetivos}
\begin{frame}
    \frametitle{Objetivos}
    \begin{block}{Objetivos Generales}
        Implementar la tecnología de redes neuronales convolucionales con el fin de obtener un modelo cuya entrada sean imágenes dermatológicas y su salida sea un mapa probabilístico de las regiones dentro de ella.
        
    \end{block}
    \begin{block}{Objetivos Específicos}
        Determinar la secuencia de pre-procesamiento necesario para adaptar imágenes de diferentes rangos de resolución a la resolución admitida por la arquitectura de la red neuronal convolucional y verificar la confiabilidad de la predicción mediante criterios de evaluación de mapas binarios y probabilísticos.
        
    \end{block}
\end{frame}

\section{Antecedentes}

\begin{frame}
    \frametitle{Antecedentes}
    \blindtext[1]
\end{frame}

\begin{frame}
    \frametitle{Antecedentes}
    \blindtext[1]
\end{frame}


\section{Estado del Arte}
\begin{frame}
    \frametitle{Estado del Arte}
    \blindtext[1]
\end{frame}

\section{Implementación}
\begin{frame}
    \frametitle{Implementación}
    \blindtext[1]
\end{frame}

\section{Experimentos}
\begin{frame}

    \frametitle{Experimentos}
    \blindtext[1]

\end{frame}

\begin{frame}
    \frametitle{Experimentos}
    \includegraphics[width=\textwidth]{example-image-a}
\end{frame}

\section{Conclusión}
\begin{frame}
    \frametitle{Conclusión}
    \blindtext[1]
\end{frame}

\end{document}